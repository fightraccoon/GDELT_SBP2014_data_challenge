
%%%%%%%%%%%%%%%%%%%%%%% file typeinst.tex %%%%%%%%%%%%%%%%%%%%%%%%%
%
% This is the LaTeX source for the instructions to authors using
% the LaTeX document class 'llncs.cls' for contributions to
% the Lecture Notes in Computer Sciences series.
% http://www.springer.com/lncs       Springer Heidelberg 2006/05/04
%
% It may be used as a template for your own input - copy it
% to a new file with a new name and use it as the basis
% for your article.
%
% NB: the document class 'llncs' has its own and detailed documentation, see
% ftp://ftp.springer.de/data/pubftp/pub/tex/latex/llncs/latex2e/llncsdoc.pdf
%
%%%%%%%%%%%%%%%%%%%%%%%%%%%%%%%%%%%%%%%%%%%%%%%%%%%%%%%%%%%%%%%%%%%


\documentclass[runningheads,a4paper]{llncs}

\usepackage{amssymb}
\setcounter{tocdepth}{3}
\usepackage{graphicx}

\usepackage{url}
\urldef{\mailsa}\path|{alfred.hofmann, ursula.barth, ingrid.haas, frank.holzwarth,|
\urldef{\mailsb}\path|anna.kramer, leonie.kunz, christine.reiss, nicole.sator,|
\urldef{\mailsc}\path|erika.siebert-cole, peter.strasser, lncs}@springer.com|    
\newcommand{\keywords}[1]{\par\addvspace\baselineskip
\noindent\keywordname\enspace\ignorespaces#1}

\begin{document}

\mainmatter  % start of an individual contribution

% first the title is needed
\title{Discovering Bilateral and Multilateral Causal Events in GDELT}

% a short form should be given in case it is too long for the running head
\titlerunning{Lecture Notes in Computer Science: Authors' Instructions}

% the name(s) of the author(s) follow(s) next
%
% NB: Chinese authors should write their first names(s) in front of
% their surnames. This ensures that the names appear correctly in
% the running heads and the author index.
%
\author{Lei Jiang
\thanks{E-mail:lionelchange@gmail.com}%
\and Fan Mai
}
%
\authorrunning{Lecture Notes in Computer Science: Authors' Instructions}
% (feature abused for this document to repeat the title also on left hand pages)

% the affiliations are given next; don't give your e-mail address
% unless you accept that it will be published
\institute{Tapjoy Inc., San Francisco, CA 94104\\
\and 
Department of Sociology, University of Virginia \\
Charlottesville, VA 22093 USA
}

%
% NB: a more complex sample for affiliations and the mapping to the
% corresponding authors can be found in the file "llncs.dem"
% (search for the string "\mainmatter" where a contribution starts).
% "llncs.dem" accompanies the document class "llncs.cls".
%

\toctitle{Lecture Notes in Computer Science}
\tocauthor{Authors' Instructions}
\maketitle


\begin{abstract}
GDELT is one of the richest sources of event data and the hidden information could be potentially of tremendous value for its real-world reflections. While the majority of past research on these data sets puts emphasis on featured event series alone, such as Arab spring and violence in Afghanistan, this paper is focused on exploiting the causality between multiple types of events in different countries. In other words, for example, the actual impact of an event would result in responses from other parties, or even an escalation that more events will come thereafter. With the causal link in real-world context, time series analysis is performed on GDELT data to discover such causal links: we start from an exploration of bilateral and multilateral events in certain countries, where a strong correlation is shown; then Granger test is systematically conducted in the large-scale data set to help to identify causality in a rigorous manner. Further, with the addition of data in related parties, a scenario of prediction task is tested to show enhanced predictability as a validation of this methodology.
\keywords{GDELT, Granger causality, bilateral events, time series analysis, prediction}
\end{abstract}


\end{document}
